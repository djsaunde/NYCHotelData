\documentclass[useAMS, referee, usenatbib]{biom}
%documentclass[useAMS]{biom}

%%%%% PLACE YOUR OWN MACROS HERE %%%%%

\def\bSig\mathbf{\Sigma}
\newcommand{\VS}{V\&S}
\newcommand{\tr}{\mbox{tr}}

%\usepackage[figuresright]{rotating}

\usepackage{amsfonts}
\usepackage{caption}

\listfiles

\title{Unifying NYC Taxicab Records and Hotel Occupancy Data}

\author
{Daniel J. Saunders \emailx{djsaunde@cs.umass.edu} \\
College of Computer and Information Sciences, University of Massachusetts, Amherst, Massachusetts
\and
Christian Rojas \emailx{rojas@resecon.umass.edu} \\
Department of Resource Economics, University of Massachusetts, 
Amherst, Massachusetts
\and
Debi Mohapatra \emailx{dmohapatra@umass.edu} \\
Department of Resource Economics, University of Massachusetts, 
Amherst, Massachusetts}

\begin{document}

\label{firstpage}

\begin{abstract}
A method for fast distributed processing of New York City taxicab trip records with respect to a secondary dataset of daily hotel information is presented. An algorithm used to select an optimal distance threshold for capturing taxi trips relevant to predicting hotel metrics is motivated and developed. Two hotel outlier removal techniques based on distribution matching and simple outlier detection are discussed, and used to reduce prediction error. We show visualizations of taxicab trips in NYC in relation to hotel locations. We investigate the effect of predicting daily hotel occupany rates and pricing with and without conditioning on daily nearby taxi traffic. Potential applications of the taxi trip records database and other data sources are discussed as avenues of potential future work.
\end{abstract}

\begin{keywords}
Applied economics; Data analysis; Distributed computing.
\end{keywords}

\maketitle

\section{Introduction}
\label{s:intro}

The availability of various forms of empirical commercial and industrial data is crucial for the solution of real-world economics problems. At the scale of cities and nations, however, certain data present major data processing challenges. With the advent of massively parallel and relatively cheap distributed computing, however, methods for the processing and analysis of such datasets are not entirely out of reach.

The New York City (NYC) Taxi \& Limousine Commission (TLC) Trip Record Data consists in part of yellow and green taxicab trip records spanning the years 2009-2017. Recorded attributes include pick-up and drop-off date and time of day, pick-up and drop-off geospatial or zone-coded location, trip distances, fares, and others. Careful exploration and analysis of such a massive observational dataset may result in powerful insights in transportation issues in NYC. Methods developed on this particular dataset may be generalized to other urban settings in which such data is or becomes available.

Although interesting studies have been done on the NYC taxi records data, combining it with additional relevant datasets should allow researchers to propose and test more powerful economic hypotheses. We study a unification of the taxi data with the occupancy rates of 178 NYC hotels from the years 2013 to 2016. We are interested in trips which \textit{begin or end within distance} $d$ of at least one hotel in the dataset. To find the best choice of $d$ over all hotels, we demonstrate an optimization procedure motivated from a simple distribution matching approach. Hotels with abnormal amounts of taxicab traffic may be eliminated from consideration during or after the optimization process.

We hypothesize that both hotel occupancy rates and pricing can be predicted from the density of daily nearby taxicab pick-up and drop-off rides, combined with other relevant attributes. We hypothesize that hotel occupancy and, indirectly, pricing may be more accurately predicted as a function of daily nearby taxi ride density along with more generic attributes than without the nearby taxi rides information.

\section{Related Work}
\label{s:related}

An analytics model is proposed in \citet{Ferreira2013VisualEO} which allows users to visually query taxi trips. Standard queries about the data are supported, as well as spatio-temporal ``origin-destination'' queries, allwoing users to discover mobility patterns throughout the city. NYC hotspots are identified in \citet{Stoyanovich2017ZoomingIO}, in which taxi trips are represented as straight-line trajectories from pick-up to drop-off coordinates. Hotspots are used to identify lack of convenient public transportation options, and to suggest the addition of bus routes and ride-sharing options. A custom software approach is developed on top of the distributed processing system Apache Spark, allowing efficient analysis of taxi trajectory graphs as they evolve. In another work, a visual analytics method is proposed to study urban mobility patterns using graph modeling of taxicab trajectories \citep{Huang2016TrajGraphAG}. Dynamics of Shenzhen taxis is analyzed using graph analysis, and is studied at multiple scales by using a graph partitioning algorithm to produce region-level visual analytics. Users of the system can interactively explore street-level city traffic patterns. 

A technique for event-guided exploration of large spatio-temporal data is introduced in \citet{Doraiswamy2014UsingTA}, on the basis that manual exploration of such data is time-consuming and ineffective. Computational topology is used to discover interesting events in the data, and an algorithm is developed to group and index events which can be interactively explored or queried by users. This approach is validated on NYC taxi trips as well as subway service records. An analysis of the NYC taxi trips database in combination with various relevant urban data is presented in \citet{Wu2016InterpretingTD}. Correlations between taxi and the external datasets are used to the predict traffic dynamics; namely, it it shown that points-of-interest (POIs) can predict regular traffic patterns, geo-tagged tweets can explain traffic caused by atypical events, and weather data may explain abnormal drops in traffic density. \citet{Chu2014VisualizingHT} develop a method to discover and analyze information contained in a massive dataset of taxi trajectories. Geospatial coordinates are replaced with traversed street names, which are subsequently modeled with topic modeling tools. The ``hidden themes'', or the discovered topics from the employed topic models, are used to analyze mobility patterns with a visual analytics system. 

The efficient processing of spatio-temporal datasets (e.g, NYC taxi trip records) often requires specialized software and hardware. A novel indexing scheme over spatio-temporal data is developed for use with general-purpose graphics processing units (GPUs) in \citet{Doraiswamy2016AGI}. The index allows sub-second query speeds over large spatio-temporal datasets such as the NYC taxi data, and a massive number of tweets collected from Twitter.

\section{Methods}
\label{s:methods}

\subsection{Data processing}

The dataset of hotel occupancy rates contains information from January 2013 until present day. The dataset of NYC hotel taxi records contains data from January 2009 until present day; however, geospatial coordinates of pick-up and drop-off locations are only recorded from January 2009 until June 2016. Given these constraints, our experiments consider only those records which lie in the date range of January 2013 until June 2016, approximately 3.5 years of joint data. Our analyses are concerned only with the yellow and green taxi services in NYC, and we discard all data concerning ``for-hire vehicle'' services (e.g., Uber, Lyft, and others).

We use a supercomputing cluster in which users are provisioned with a maximum of 40 CPU compute nodes. On each node, a Python program is dispatched to \textit{pre-process} a single month's worth of taxicab trip data with respect to the set of NYC hotels of interest. In particular, upon specifying a \textit{distance criterion} $d$, each process will extract trips from its month worth of data which begin or end with $d$ feet of any of the hotels under consideration. Geospatial distance calculations are accomplished using Vincenty's formula \citep{bessel_calculation_2010}. Loading and pre-processing of the taxi trip records is accomplished using the Dask parallel computation and task scheduling Python library \cite{dask}.

Prior to pre-processing, each month of yellow and green taxi trip data amounts to approximately 2.5Gb and 300Mb, respectively, totalling $\approx$115Gb over the course of the experimental date range. Pre-processing the dataset using a distance criterion $d = 300$ feet reduces this to two files of size $\approx$20Gb each, one of trips beginning near hotels of interest (\textit{destinations}), and another ending nearby (\textit{origins}).

Pre-processing a month of yellow taxi data using the aforementioned Python program and distance criterion $d = 300$ requires \textit{at most} 40 minutes and 50Gb of random-access memory. Using the super-computing cluster with a 40-node per-user allotment, processing all 42 months of taxi trip date requires only $\approx$80 minutes. If equipped with more than 42 nodes, the processing time would be cut in half.

\subsection{Distance criterion optimization}

A crucial assumption in the analyses of the taxi dataset is that taxi trips that originate or culminate near a hotel is likely to indicate that a guest is arriving or leaving the hotel. With this in mind, we can use the locations to which the guest travels, or where the guest has traveled from, to draw conclusions about typical patrons per hotel.

The distance criterion $d$ should be selected such that it maximizes economic predictability in some sense. We compare a dataset of hotel occupancy information with proportions of nearby taxi trips, per hotel. We are interested in finding a setting of $d$ such that some divergence measure $\mathcal{D}$ between empirically observed distributions $\hat{p}_{\mathrm{occ}}$ and $\hat{p}_{\mathrm{taxi}}$ is minimized. We consider relative entropy $\left( \sum_x \hat{p}_{\mathrm{occ}} (x) \log \frac{\hat{p}_{\mathrm{occ}} (x)}{\hat{p}_{\mathrm{taxi}} (x)} \right)$, summed absolute differences $\left( \sum_x | \hat{p}_{\mathrm{occ}} (x) - \hat{p}_{\mathrm{taxi}} (x) | \right)$, and summed relative differences, in which the smaller-valued proportion is used as the base in the quotient $\left( \sum_x \frac{\min \{ \hat{p}_{\mathrm{taxi}} (x), \hspace*{0.1cm} \hat{p}_{\mathrm{occ}} (x) \}}{\max \{ \hat{p}_{\mathrm{taxi}} (x), \hspace*{0.1cm} \hat{p}_{\mathrm{occ}} (x) \}} \right)$. Note that the divergence measures used are not necessarily true $f$-divergences, but were selected as a result of the properties combined taxi trip records and hotel occupany datasets. 

After pre-processing the NYC taxi data using a suitably large distance criterion (e.g., $d = 300$ft.), a range of candidate distance criterions are considered (e.g., $d \in [25 \mathrm{ft.}, 50 \mathrm{ft.}, ..., 275 \mathrm{ft.}, 300 \mathrm{ft.}]$). The empirical per-hotel proportions of nearby pick-up and / or drop-off taxi trips $\hat{p}_{\mathrm{taxi}}$ is computed from the data for each choice of $d$, and we calculate the selected divergence measure with respect to the fixed $\hat{p}_{\mathrm{occ}}$; that is, the observed per-hotel proportions of rented rooms over all hotels over the same time period. The criterion $d$ giving the minimal divergence value is said to be the ``best'', and the dataset obtained from this choice of $d$ can be used for downstream analyses.

\subsection{Removing outlier hotels}

Certain hotels under consideration may have an abnormal large or small number of nearby taxicab pick-ups or drop-offs depending on their location in the city. For example, Hotel Pennsylvania, located adjacent to Pennsylvania station, has a disproportionately large number of both pick-up and drop-off taxi trips due in part to the traffic that the train station incurs. This observation makes it difficult to justify using such hotels in any joint hotel occupancy-taxicab ride distribution analysis.

Using the same divergence measures as listed above, we propose two methods for the removal of outlier hotels:

\begin{itemize}

\item [1.] Iteratively remove hotels from the occupancy dataset during the distance optimization process until the divergence measure between distributions, $\mathcal{D} (\hat{p}_{\mathrm{occ}} || \hat{p}_{\mathrm{taxi}})$, is below some threshold. Picking the threshold is a hyper-parameter choice, and may be difficult to select \textit{a priori}.

\item [2.] After selecting a distance criterion $d$, apply a standard outlier-detection algorithm based on the per-hotel divergence measure values. \textbf{TODO}

\end{itemize}

\subsection{Predicting hotel occupancy and pricing}

We establish a simple estimation baseline using only information from the hotel occupancy dataset to predict daily hotel room demand. In particular, we fit an ordinary least squares (OLS) regression model to the hotel occupany data, in which we assume the room demand is a linear function of the hotel identity, the day of the week, and the date (\textbf{TODO}: should the baseline include the year, month, day, and day of week instead?). This information allows the model to discriminate typical per-hotel room demand rates, as well as to capture the weekly and yearly trends in occupancy rates. A simple multi-layer perceptron (MLP) is also fit to this data, with parameters (hidden layer sizes and regularization constant) selected by random hyper-parameter search. This model is able to learn non-linear interactions between the three chosen observational features. The data is partitioned into 80\% train, 10\% validation, and 10\% test, and $R^2$ and mean-squared error (MSE) of both models are reported in \ref{s:results}.

To evaluate the effect of conditioning hotel occupany predictions on available nearby taxi trip densities, we include both nearby pick-up and drop-off trips as separate features in both the OLS and MLP models, as described above. Hyper-parameters are re-selected according to a random hyper-parameter search, and the dataset is partitioned into train, validation, and test subsets as before. We train and test models using a range of nearby taxi distance thresholds $d \in \{25 \mathrm{ft.}, 50 \mathrm{ft.} ..., 300 \mathrm{ft.} \}$, in order to validate the choice of distance criterion established by the optimization method described above.

\section{Results}
\label{s:results}

\subsection{Outlier removal}

Using the first outlier removal method, as part of the distance criterion optimization procedure, we present tables of the order of hotel removals from the dataset for each divergence measure. Included also are the removed hotel's value of the calculated divergence measure, as well as the optimal distance criterion $d$ for that particular iteration. Though there are 178 hotels, we truncate each table after just 10 iterations in order to give an idea of the removal behavior given a particular divergence measure.

\begin{table}
\caption{Order of hotel removals using the relative divergence divergence measure. Optimal distance criterion, per-hotel observed occupancy and nearby taxi trip proportions, and calculated divergence value also included.}
\label{t:relative_divergence}
\begin{center}
 \begin{tabular}{||c|c|c|c|c|c||}
 \hline
 Removal & Hotel & $\hat{p}_{\mathrm{occ}}$ & $\hat{p}_{\mathrm{taxi}}$ & Best $d$ & Rel. diff. \\ [0.5ex] 
 \hline
 1 & Res. Inn ... Trade Center & $6 \times 10^{-7}$ & $2.6 \times 10^{-3}$ & 300 & 4,549 \\
 2 & Hotel Gansevoort & $3 \times 10^{-3}$ & $2 \times 10^{-2}$ & 190 & 6.33 \\
 3 & Courtyard ... Herald Square & $2.2 \times 10^{-3}$ & $1.3 \times 10^{-2}$ & 180 & 5.93 \\
 4 & Hilton ... Park Avenue & $5.7 \times 10^{-4}$ & $3.3 \times 10^{-3}$ & 180 & 5.74 \\
 5 & Hotel On Rivington & $1.7 \times 10^{-3}$ & $9.6 \times 10^{-3}$ & 195 & 5.62 \\
 6 & Holiday Inn Express ... & $4.3 \times 10^{-3}$ & $8 \times 10^{-4}$ & 195 & 5.28 \\
 7 & Hilton New York Midtown & $3.6 \times 10^{-2}$ & $7.4 \times 10^{-3}$ & 195 & 4.91 \\
 8 & Doubletree ... Fin. District & $7 \times 10^{-3}$ & $1.6 \times 10^{-3}$ & 190 & 4.27 \\
 9 & Sohotel & $9 \times 10^{-4}$ & $3.8 \times 10^{-3}$ & 190 & 4.2 \\
 10 & Res. Inn ... Central Park & $2 \times 10^{-3}$ & $8.1 \times 10^{-3}$ & 180 & 4.02 \\
 \hline
\end{tabular}
\end{center}
\end{table}

\subsection{Predicting hotel occupancy and pricing}

\textbf{TODO}

\section{Conclusion}
\label{s:conclusion}

\backmatter

\section*{Acknowledgements}

The authors would like to thank the College of Computer and Information Sciences for the gracious allowance of supercomputing time.

\bibliographystyle{biom} 
\bibliography{nyctaxi}

\label{lastpage}

\end{document}