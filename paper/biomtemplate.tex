\documentclass[useAMS,referee]{biom}
%documentclass[useAMS]{biom}

%%%%% PLACE YOUR OWN MACROS HERE %%%%%

\def\bSig\mathbf{\Sigma}
\newcommand{\VS}{V\&S}
\newcommand{\tr}{\mbox{tr}}

%\usepackage[figuresright]{rotating}

%%%%%%%%%%%%%%%%%%%%%%%%%%%%%%%%%%%%%%%%%%%%%%%%%%%%%%%%%%%%%%%%%%%%%

\title{Unifying NYC Taxicab Records and Hotel Occupancy Data}

\author
{Daniel J. Saunders \emailx{djsaunde@cs.umass.edu} \\
College of Computer and Information Sciences, University of Massachusetts, Amherst, Massachusetts
\and
Christian Rojas \emailx{rojas@resecon.umass.edu} \\
Department of Resource Economics, University of Massachusetts, 
Amherst, Massachusetts}

\begin{document}

\label{firstpage}

\begin{abstract}
A method for fast distributed processing of New York City taxicab trip records in relation to a secondary dataset of hotel information is presented. An algorithm used to select an optimal distance threshold for capturing relevant trip records is developed. Predictions of hotel occupancy rates are conditioned on relevant taxi records, showing an improvement over a baseline predictive model. We show exploratory visualizations and analysis of taxicab trip records. We suggest and preliminarily investigate applications of the New York City taxicab trip records data and its unification with hotel occupancy information.
\end{abstract}

\begin{keywords}
Applied economics; Data analysis; Distributed computing.
\end{keywords}

\maketitle

\section{Introduction}
\label{s:intro}

The availability of various forms of empirical commercial and industrial data is crucial for the evaluation and solution of real-world economics problems. At the scale of cities and nations, however, certain data present major data processing challenges. With the advent of massively parallel and relatively cheap distributed computing, however, methods for the processing and analysis of such datasets are not entirely out of reach.

The New York City Taxi \& Limousine Commission Trip Record Data consists in part of yellow and green taxicab trip records spanning the years 2009-2017. Recorded attributes include pick-up and drop-off date and time of day, pick-up and drop-off geospatial or zone-coded location, trip distances, fares, and more.

Combining the NYC taxicab trip records data with additional relevant datasets may allow for more interesting economic conclusions. We study a unification of the taxi data with the occupancy rates of 178 NYC hotels from 2013 - 2016. We are interested in those taxicab trips which \textit{begin or end within distance} $d$ of at least one hotel in the dataset. To find the best choice of $d$ over all hotels, we create an optimization procedure motivated from a simple distribution matching procedure. Hotels with abnormal amounts of taxicab traffic are eliminated from the data during the optimization process.

We hypothesize that both hotel occupancy rates and pricing can be predicted from daily nearby taxicab pick-up and drop-off distributions. We hope that hotel occupancy and (indirectly) pricing will be a simple stochastic function of the proportion of nearby taxicab density. Indeed, our data analysis demonstrates this on a subset of hotels within a tolerable error.

\section{Related Work}
\label{s:related}

\textbf{TODO}.

\section{Methods}
\label{s:methods}

\section{Discussion}
\label{s:discuss}

\backmatter

\section*{Acknowledgements}

\section*{Supplementary Materials}

Web Appendix A, referenced in Section~\ref{s:model}, is available with
this paper at the Biometrics website on Wiley Online
Library.\vspace*{-8pt}

%  Here, we create the bibliographic entries manually, following the
%  journal style.  If you use this method or use natbib, PLEASE PAY
%  CAREFUL ATTENTION TO THE BIBLIOGRAPHIC STYLE IN A RECENT ISSUE OF
%  THE JOURNAL AND FOLLOW IT!  Failure to follow stylistic conventions
%  just lengthens the time spend copyediting your paper and hence its
%  position in the publication queue should it be accepted.

%  We greatly prefer that you incorporate the references for your
%  article into the body of the article as we have done here 
%  (you can use natbib or not as you choose) than use BiBTeX,
%  so that your article is self-contained in one file.
%  If you do use BiBTeX, please use the .bst file that comes with 
%  the distribution.  In this case, replace the thebibliography
%  environment below by 
%
%  \bibliographystyle{biom} 
% \bibliography{mybibilo.bib}

\begin{thebibliography}{}

\bibitem{ } Cox, D. R. (1972). Regression models and life tables (with
discussion).  \textit{Journal of the Royal Statistical Society, Series B}
\textbf{34,} 187--200.

\bibitem{ }  Hastie, T., Tibshirani, R., and Friedman, J. (2001). \textit{The 
Elements of Statistical Learning: Data Mining, Inference, and Prediction}.
New York: Springer.

\end{thebibliography}

\appendix

%  To get the journal style of heading for an appendix, mimic the following.

\section{}
\subsection{Title of appendix}

Put your short appendix here.  Remember, longer appendices are
possible when presented as Supplementary Web Material.  Please 
review and follow the journal policy for this material, available
under Instructions for Authors at \texttt{http://www.biometrics.tibs.org}.

\label{lastpage}

\end{document}
